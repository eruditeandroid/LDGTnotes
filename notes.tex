\documentclass{amsart}

\usepackage{style}

\title{Low Dimensional Gauge Theory}
\author{Notes by John S.\ Nolan, speakers listed below}

\begin{document}

\maketitle

\tableofcontents

\section{9/16 (Constantin Teleman) -- Low Dimensional Gauge Theory for Finite Groups and Boundary Conditions}

\subsection{$1$- and $2$-dimensional TQFT}

Let's start by considering topological quantum mechanics, i.e.\ $1$-dimensional TQFT.
This is determined by a vector space of states $\Hc$.
We may also consider a super vector space $\Hc^0 \oplus \Hc^1$
Here $S^1$ is assigned to $\dim \Hc$.
In particular, $\dim \Hc$ must be finite.

In $2$-dimensions, a TQFT is generated by a $\CC$-linear category.
We may consider various categories at various levels of sophistication:
\begin{itemize}
  \item Finite dimensional modules over a finite dimensional algebra.
  \item $\Dbcoh(X)$ for some algebraic variety $X$.
  \item $\Db(A)$ for some dga $A$.
\end{itemize}

The dualizability conditions here are rather strict:

\begin{thm}
  If a $\CC$-linear category $\Cc$ with finite colimits\footnote{in the $1$-categorical sense, i.e.\ we don't think about derived categories.} generates a $2$d TQFT, then $\Cc$ is finite semisimple.
\end{thm}

Given a finite dimensional algebra $A$, let's consider $A$ as an object of the $2$-category of algebras, bimodules, and intertwiners.
Equivalently, we replace $A$ by the category $\Mod(A)$ of finitely presented $A$-modules.
Then $A$ is always $1$-dualizable.
Furthermore, $A$ is $2$-dualizable if and only if:
\begin{itemize}
  \item $A$ is dualizable as a $\CC$-module, i.e.\ finite-dimensional.
  \item $A$ is dualizable as an $(A, A)$-bimodule, i.e.\ the functor $\Vect \to (A, A)\dBimod$ has left and right adjoints.
\end{itemize}

The bordism $2$-category has:
\begin{itemize}
  \item Objects are points.
  \item $1$-morphisms are $1$-manifolds.
  \item $2$-morphisms are surfaces.
\end{itemize}

\begin{thm}
  A category $\Cc$ is $2$-dualizable if and only if it defines a functor out of the $2$-framed bordism $2$-category.
\end{thm}

\subsection{Symmetries from groups}

In quantum mechanics, let's suppose $G$ is a finite group and we have an action $G \curvearrowright \Hc$.
We may view $\Hc$ as a module over the group ring $\CC[G]$.
By Maschke's theorem, we may write $\CC[G] = \oplus_V \End(V)$.
The algebra $\CC[G]$ therefore generates a $2$d TQFT.

This TQFT is pure $G$-gauge theory $Z_G$, and it is defined on unoriented manifolds.
The theory $Z_G$ quantizes the classical theory of $G$-bundles.
We may compute:
\begin{itemize}
  \item $Z_G(\pt) = \CC[G]\dMod$.
  \item $Z_G(S^1) = \CC[G]^G$
  \item For a closed surface $\Sigma$, $Z_G(\Sigma) = \sum_{[P] \textrm{ on } \Sigma} 1 / \card{\Aut(P)}$.
  \item For a surface with boundary $\Sigma'$, $Z_G(\Sigma')$ is a matrix with coefficients
    counting weighted numbers of extensions of bundles from the boundary to the bulk.
\end{itemize}
Counts may be understood using push-pull operations along the moduli stacks / groupoids $\Bun_G(\Sigma)$.

Given $\tau \in H^2(BG; \CC^\times)$, we may twist $\CC[G]$ to get a new algebra $\CC^\tau[G]$ (isomorphic to $\CC[G]$ as vector spaces) where
\[
  g \cdot h = \tau(g, h) gh.
\]
Alternatively, the $2$-cocycle corresponds to an associative multiplication on a line bundle $\Lsc$ over $G$, and $\CC^\tau(G) = \Gamma(G; \Lsc(\tau))$.

This new TQFT $Z_G^\tau$ requires orientations.
It gives a new way of counting bundles:
\begin{itemize}
  \item The twisted theory now has
  \[
    Z_G^\tau(\Sigma) = \sum_P \frac{1}{\card{\Aut(P)}} \int_{\Sigma} c_P^*(\tau)
  \]
  where $c_P: \Sigma \to BG$ classifies $P$.
  We need to use the orientation to perform this integration.
  \item On $S^1$, we think of the integrals $\int_{S^1} c_P^*(\tau)$ as living in $H^1(B\Aut_P; \CC^\times)$ (since we are integrating elements of $H^2_{\Aut_P}(S^1; \CC^\times)$).
  The vector spaces $H^1(B\Aut_P; \CC^\times)$ give a line bundle over the stack $G /_\Ad G$, and the (adjoint-invariant) global sections of this line bundle are $Z_G^\tau(S^1)$.
  In other words, $Z_G^\tau(S^1) = (\CC^\tau[G])^G$, with multiplication coming from the pair of pants.
\end{itemize}
The category is $CC^\tau[G]\dMod$, the category of $\tau$-projective representations of $G$.

Every $G$-representation $V$ defines a boundary theory for $Z_G$.
That is, it gives an extension of $\Bord_2 \to \Cat$ to $\Bord_2^{\textrm{colored}} \to \Cat$,
where $\Bord_2^{\textrm{colored}}$ allows for manifolds with free boundary components labeled by $V$.
This alters our counts: we take principal $G$-bundles on surfaces $\Sigma$ as usual, but we multiply the count by the trace of the holonomy on $V$.
In particular:
\begin{itemize}
  \item The cylinder with one free boundary component and one outgoing boundary component gives the map $\CC \to \CC[G]^G$ corresponding to the character $\chi_V$ of $V$. 
  \item Following this cylinder in reverse is integration against $\chi_V$.
\end{itemize}
A paper of Moore and Segal on ``open--closed theories and boundary conditions'' explains this material well.

\subsection{What do we assign to points?}

For pure $G$-gauge theory, counting bundles over a point corresponds to integrating the unit over $BG$.
Here the ``unit'' is the category $\Vect$, and there are two things we could mean by integration: limits or colimits.
``Ambidexterity'' will ensure that these agree, which is necessary to make sure we have a well-defined TQFT.

\begin{dfn}
  An action of $G$ on a linear category $\Cc$ consists of:
  \begin{itemize}
    \item An assignment sending every $g \in G$ to a linear functor $\phi_g: \Cc \to \Cc$
    \item An associativity isomorphism $\alpha_{g,h}: \phi_g \circ \phi_h \simto \phi_{gh}$
  \end{itemize}
  such that the two ways of computing $\phi_{ghk}$ agree (witnessed by a suitable commutative diagram).
\end{dfn}

We'll first try to understand limits.

\begin{dfn}
  Given $G \curvearrowright \Cc$, let 
  \[
    \Cc^G = \bset{(x \in \Cc, \{\phi_{g,x}\}_{g \in G})}{x \in \Cc, \phi_{g,x}: x \simto gx \textrm{ making the expected triangle commute.}}
  \]
  with morphisms given by the subsets of $\Hom$ ``commuting with the $G$-action.''
\end{dfn}

\begin{ex}
  Let $G$ act trivially on $\Vect$.
  Then $\Vect^G = \Rep(G)$.
\end{ex}

\begin{ex}
  Let $\tau \in H^2(BG; \CC^\times)$.
  Let $G$ act trivially objectwise on $\Vect$, but let the associators be defined by $\tau$.
  Then $\Vect^{G,\tau} = \Rep^\tau(G)$.
\end{ex}

\begin{ex}
  Let $\Cc = \Vect[G]$, the category of vector bundles over $G$.
  Let $G$ act on $\Cc$ by translation.
  Then $\Vect[G]^G= \Vect$.
\end{ex}

Now we move on to colimits, which can be understood as crossed products.

\begin{dfn}
  Let $G \curvearrowright \Cc$.
  Define $\Cc_G$ as the category constructed as follows:
  \begin{itemize}
    \item Objects are objects of $\Cc$.
    \item Add new morphisms $\phi_{x,g}: x \to gx$ satisfying the expected commutative triangle.
      These should also satisfy commutative squares with the old morphisms.
  \end{itemize}
\end{dfn}

\begin{ex}
  Let $\Cc$ be the category with one object with endomorphisms $\CC$.
  Let $G$ act trivially on $\Cc$.
  Then $\Cc_G$ is the category with one object with endomorphisms $\CC[G]$.
  We can use this to show that $\Vect_G$ is the category of free $\CC[G]$-modules.
  Completing this under finite colimits gives $\Rep(G) = \Vect^G$.
\end{ex}

\section{9/23 (Constantin Teleman) -- Discrete Model TQFTs}

Today we'll talk more about discrete models for TQFT in dimensions 2 and 3.
We'll mention electromagnetic duality, which corresponds to discrete Langlands duality and relates to Koszul duality for algebras.

Fix a finite group $G$ and a class $\tau \in H^3(BG; \CC^\times)$.
The corresponding gauge theory $\Tc_{G,\tau}$ can be understood in a few ways:
\begin{itemize}
  \item $\Tc_{G,\tau}(\pt)$ is an object of a 3-category.
    The duality operations on $\Tc_{G,\tau}(\pt)$ allow us to construct invariants for manifolds via handle decompositions.
  \item $\Tc_{G,\tau}$ also gives manifold invariants by direct computations (summing over all fields).
    At the top level (3-manifolds), this is a literal sum.
    In lower dimensions, this is a categorical sum.
\end{itemize}

\subsection{Fusion categories}

Specifically, our ambient 3-category is the 3-category of \emph{fusion categories}, i.e.\ finite semisimple linear categories with associative tensor products and internal left / right duals.
Sometimes people also require the unit $\onebb$ to be simple, i.e.\ $\End(\onebb) = \CC$.
The existence of internal duals corresponds to the existence of a nonzero topological boundary condition for the theory.

\begin{ex}
  Some examples of fusion categories are:
\begin{itemize}
  \item $(\Vect, \otimes)$, the symmetric monoidal category of $\CC$-vector spaces.
  \item $M_n(\Vect)$, the monoidal category of $\Vect$-valued matrices.
  \item $\Vect\angles{G}$, the monoidal category of $G$-graded vector spaces (with convolution as operation).
    This may be thought of as the category of vector bundles on $G$.
  \item $\Vect^\tau\angles{G}$, a version of $\Vect\angles{G}$ with associator given by $\tau_{g,h,k}$.
    The cocycle condition on $\tau$ ensures that the associator is consistent.
\end{itemize}
\end{ex}

The 1-morphisms $F_1 \to F_2$ in our 3-category are finite semisimple bimodule categories ${}_{F_1} M_{F_2}$.
Composition is given by the relative tensor product:
\[
  {}_{F_1} M_{F_2} \circ {}_{F_2} N_{F_3} = {}_{F_1} M \boxtimes_{F_2} N_{F_3},
\]
where ${}_{F_1} M \boxtimes_{F_2} N_{F_3}$ is defined by the universal property
\[
  \Lin(M \boxtimes_{F_2} N, P) = \Bilin_{F_2}(M \times N; P)
\]
for a suitable definition of ``bilinear functor.''
The following nontrivial result ensures that composition is well-defined.
\begin{thm}
  If $F_2$ is a fusion category, then $\boxtimes_{F_2}$ sends finite semisimple bimodule categories to finite semisimple bimodule categories.
\end{thm}

Ostrik's principle (a version of the Barr--Beck theorem) allows us to reduce the category level of some calculations.

\begin{thm}[Ostrik]
  Let $F$ be a fusion category, $M$ a (finite semisimple) $F$-module, and $m \in M$ a generating object (after incorporating the $F$-action).
  Then $M$ is equivalent to the category $F_a$ of right $a$-modules in $F$, where $a = \ul{\End}_F(m)$ (so $\Hom_F(x, a) = \Hom_M(x \cdot m, m)$ for all $x \in M$).
\end{thm}

The equivalence $F_a \simto M$ is given by $x \mapsto x \cdot_a m$.

\begin{ex}
  Let $F = \Vect\angles{G}$.
  Semisimple module categories over $F$ are the same as semisimple linear categories with $G$-action.
  Every such module category is a sum of indecomposables (by definition).
  In every such indecomposable, the isomorphism classes of simples are $G / H$ for some $H \subset G$.
  We may also twist the action by $\tau \in H^2(BH; \CC^\times)$.
  Every indecomposable module category is then $\Vect^\alpha\angles{G/H}$ for some $H \subset G$ and some $\alpha \in H^2(BH; \CC^\times)$.

  We may try to work out examples of Ostrik's principle in this context.
  For $M = \Vect\angles{G}$ (the regular module), we may take $m = \onebb$ and then $a = \onebb$.
  For $M = \Vect$ (with trivial action), we may take $m = \CC$ and then $a = \sum_{g \in G} \CC \cdot g$, the group ring.
\end{ex}

\begin{ex}
  If $F = \Vect^\tau\angles{G}$, then indecomposable module categories correspond to:
  \begin{itemize}
    \item Subgroups $H \subset G$,
    \item Cochains $\alpha \in C^2(BH; \CC^\times)$ satisfying $\delta \alpha = \tau$, all modulo coboundaries.
  \end{itemize}
\end{ex}

Here are some useful facts about classification:
\begin{itemize}
  \item ``Isomorphisms'' of fusion categories (in our 3-category) correspond to categorical Morita equivalences.
  That is, we have $P: F_1 \simto F_2$ if $P \boxtimes_{F_2} P^\vee \simeq F_1$ and $P^\vee \boxtimes_{F_1} P \simeq F_2$.
  Any fusion category $F$ is Morita equivalent to one with simple $\onebb$.
  \item Any finite semisimple module category is a sum of indecomposables.
  \item Given a fusion category $F$, there are only finitely many isomorphism classes of module categories.
  \item (Ostrik) If $a \in F$ is a separable algebra (i.e.\ there exists a splitting $s: a \to a \otimes a$ of the multiplication as $(a, a)$-bimodules), then $F_a$ is a semisimple category.
\end{itemize}

\subsection{Constructing $G$-gauge theory by counting bundles}

The upshot is that $G$-gauge theory is controlled by maps to $BG$.
For low-dimensional manifolds, we need to perform a ``categorical summation'' over $G$-bundles.

This works for maps to any space $X$ with finite homotopy groups.
We may start by defining $\int_X \CC = \Gamma(X, \CC)$ the space of flat sections.
More generally, for $\tau \in H^1(X; \CC)$, we define $\int_X \CC_\tau = \Gamma(X, \CC_\tau)$.
If we've defined $\int_X \onebb_n$ where $\onebb_n$ is the unit $n$-category, then we may define 
\[
  \int_X \onebb_{n+1} = \oplus_P \int_{\Omega_P X} \onebb_{n+1}.
\]
This is an algebra and we get an $(n+1)$-category of modules.
More generally we may incorporate twists into the story.
Ambidexterity (allowing us to view this colimit as a limit) is conjectured but not written up.

For $G$-gauge theory, we have:
\begin{itemize}
  \item $\pt$ is sent to $\Map(pt, BG) = BG$, which quantizes to $\Vect\angles{G}$.
  \item $S^1$ is sent to $\Map(S^1, BG) = G /_{\Ad} G$, which quantizes to $\Vect^G\angles{G} \simeq \oplus_{[g]} \Rep(Z_g)$.
    This category is $\int_{G /_{\Ad} G} \Vect$.
    The pair of pants multiplication makes $\Vect^G\angles{G}$ into a braided tensor category (the Drinfeld center of $\Vect\angles{G}$).
    The operation here is given by push-pull along the pair of pants.
\end{itemize}

\subsection{Boundary conditions}

\begin{thm}
  Every fusion category $F$ defines a TQFT for framed 3-manifolds.
  This sends $S^1$ with the blackboard framing to $Z(F) = \End_{(F,F)}(F)$.
  The other framing is sent to the ``cocenter'' $F \boxtimes_{F \boxtimes F\op} F$.
\end{thm}

Conjecturally, if the fusion category $F$ is spherical, we may pass to oriented manifolds.
(This conjecture is known for gauge theories.)

Every finite semisimple $F$-module gives a boundary condition.
This may be understood as a functor throm the fully extended bordism category of $(0, 1, 2, 3)$-manifolds with part of the codimension $1$ boundary colored (such that the cutting is transverse to the colored boundary).
Another way to think of it is as a morphism from $\onebb$ to a generator for the bulk TQFT.

\begin{ex}
  For 3d gauge theory with boundary $H \subset G$, the theory counts $G$-bundles on the bulk with an $H$-reduction on the boundary.
\end{ex}

\section{(9/30) Chan Bae -- Fusion Categories and Homotopy Theory}

The material in today's talk can be found in ``Fusion categories and homotopy theory'' by Etingof, Nikshych, and Ostrik.

\subsection{Categorical $n$-groups}

\begin{dfn}
  A categorical $n$-group is a monoidal $n$-groupoid where all objects are invertible under the monoidal operation.
\end{dfn}

\begin{ex}
  A categorical $0$-group is an ordinary group.
\end{ex}

\begin{ex}
  A categorical $1$-group is also known as a ``gr-category'' or sometimes a ``$2$-group,'' though we will never use this terminology.
\end{ex}

\subsection{Morita equivalence of fusion categories}

Let $\Cc$ and $\Dc$ be fusion categories (over an algebraically closed field of characteristic zero).
We'll assume the unit objects are simple.

Let $\Mc$ be a $(\Cc, \Dc)$-bimodule category.
The opposite category $\Mc\op$ is naturally a $(\Dc, \Cc)$-bimodule where
\[
  d \otimes \ol{m} \otimes c = \ol{{}^* c \otimes m \otimes d^*}.
\]
\begin{prop}
  The following are equivalent:
  \begin{enumerate}
    \item $\Mc\op \boxtimes_\Cc \Mc \simeq \Dc$.
    \item $\Mc \boxtimes_\Dc \Mc\op \simeq \Cc$.
    \item The map $d \mapsto (- \otimes d)$ is an equivalence $\Dc\mop \simeq \Hom_\Cc(\Mc, \Mc)$, where $(-)\mop$ denotes the monoidal opposite.
    \item The map $c \mapsto (c \otimes -)$ is an equivalence $\Cc \simeq \Hom_\Dc(\Mc, \Mc)$.
  \end{enumerate}
\end{prop}

\begin{dfn}
  If any of the previous conditions hold, we say that $\Mc$ is \emph{invertible}.
  If such an $\Mc$ exists, we say $\Cc$ and $\Dc$ are \emph{Morita equivalence}.
\end{dfn}

\begin{dfn}
  The \emph{Brauer--Picard $3$-groupoid} $\ul{\ul{\BrPic}}$ is the $3$-groupoid with:
  \begin{itemize}
    \item Objects: fusion categories,
    \item $1$-morphisms: invertible bimodules,
    \item $2$-morphisms: bimodule equivalences,
    \item $3$-morphisms: natural isomorphisms.
  \end{itemize}
  We may truncate this to get a $2$-groupoid $\ul{\BrPic}$ and a $1$-groupoid $\BrPic$.
\end{dfn}

If we fix a fusion category $\Cc$, we get a $2$-group $\ul{\ul{\BrPic}}(\Cc)$ of Morita-automorphisms of $\Cc$.
We may again truncate to get $\ul{\BrPic}(\Cc)$ and $\BrPic(\Cc)$.
The ``Brauer--Picard group of $\Cc$'' refers to $\BrPic(\Cc)$, the group of Morita-automorphisms of $\Cc$ up to equivalence.

\begin{ex}
  $\BrPic(\Vect_k)$ is the abelian group of Morita equivalence classes of central simple algebras over $k$.
\end{ex}

\begin{thm}
  Let $\Cc$ be a fusion category.
  Then $\BrPic(\Cc)$ is a finite group.
\end{thm}

\begin{proof}
  We may view a $(\Cc, \Cc)$-bimodule $\Mc$ may be viewed as a tensor functor $\Cc \boxtimes \Cc\mop \to \End(\Mc)$.
  A combinatorial argument may be used to restrict the size of $\Mc$.
  One finishes by using Ocneanu rigidity to ensure that there may only be finitely many such tensor functors.
\end{proof}

\begin{ex}
  If $A$ is an abelian group, we may compute $\BrPic(\Vect_A) = \Orm(A \oplus A^*)$.
\end{ex}

\subsection{Drinfeld centers}

\begin{dfn}
  Let $\Cc$ be a fusion category.
  The \emph{Drinfeld center} $\Zc(\Cc)$ is the braided fusion category of $(\Cc, \Cc)$-bimodule functors $\Cc \to \Cc$.
\end{dfn}

\begin{dfn}
  Write $\ul{\EqBr}$ for the groupoid of braided fusion categories and braided equivalences.
\end{dfn}

\begin{thm}
  There is an equivalence $\ul{\BrPic}(\Cc) \simeq \ul{\EqBr}(\Zc(\Cc))$.
\end{thm}

\begin{proof}[Sketch of proof]
  To define a map $\ul{\BrPic}(\Cc) \to \ul{\EqBr}(\Zc(\Cc))$, let $\Mc_\Cc$ be an indecomposable $\Cc$-module.
  Let $\Cc_\Mc^* = \Hom_\Cc(\Mc, \Mc)$.
  Then $\Mc$ is a left $\Cc_\Mc^* \boxtimes \Cc\mop$-module, and $\End_{\Cc_\Mc^* \boxtimes \Cc\mop}(\Mc)$ may be identified with both:
  \begin{itemize}
    \item (left multiplication by) objects of $\Zc(\Cc_\Mc^*)$, and
    \item (right multiplication by) objects of $\Zc(\Cc)$.
  \end{itemize}
  This gives $\Zc(\Cc) \simeq \Zc(\Cc_\Mc^*)$.
  If $\Mc$ is invertible, then $\Cc_\Mc^* \simeq \Cc$, so we may identify endofunctors of $\Mc$ with (right multiplication by) objects of $\Cc$.
  This gives $\Zc(\Cc_\Mc^*) \simeq \Zc(\Cc)$, and thus a composite $\Zc(\Cc) \to \Zc(\Cc_\Mc^*) \to \Zc(\Cc)$.
  One may check that this is a braided autoequivalence.

  To go in the reverse direction (from $\ul{\EqBr}(\Zc(\Cc))$ to $\ul{\BrPic}(\Cc)$), consider the forgetful functor $F: \Zc(\Cc) \to \Cc$.
  This has a (left and right) adjoint $I: \Cc \to \Zc(\Cc)$, and $I(\onebb)$ is a commutative (really $\EE_2$-)algebra object in $\Zc(\Cc)$.
  If $\alpha$ is our braided automorphism, then $L = \alpha\inv(I(\onebb))$ is a commutative algebra in $\Zc(\Cc)$.
  We may write $\Cc \simeq \Mod_L(\Zc(\Cc))$.
  Decompose $L = \oplus_i L_i$, and we may map $\ul{\EqBr}(\Zc(\Cc))$ to $\ul{\BrPic}(\Cc)$ by sending $\alpha$ to $\Mod_{L_i}(\Cc)$.
  This turns out to be independent of the choice of $L_i$.
\end{proof}

\begin{thm}
  $\Zc$ gives a fully faithful embedding $\ul{\BrPic} \to \ul{\EqBr}$.
\end{thm}

\subsection{Group extensions}

Let $G$ be a finite group.

\begin{dfn}
  A \emph{$G$-grading} on $\Cc$ is a decomposition $\Cc = \oplus_{g \in G}$, where:
  \begin{itemize}
    \item Each $\Cc_g$ is a full abelian subcategory, and
    \item $x_g \otimes x_h \in \Cc_{gh}$.
  \end{itemize}
\end{dfn}

This implies $\Cc_e$ is a full fusion subcategory and each $\Cc_g$ is a $(\Cc_e, \Cc_e)$-bimodule.
Any two gradings admit a common refinement, so any $\Cc$ admits a \emph{universal grading group} $U_\Cc$.

\begin{prop}
  The multiplication functor $M_{g,h}: \Cc_g \boxtimes \Cc_h \to \Cc_{gh}$ is a bimodule equivalence.
\end{prop}

This gives rise to a group homomorphism $G \to \BrPic(\Cc_e)$ (where $g \mapsto \Cc_g$).

If $G$ is a categorical $n$-group, there is a classifying space $BG$.
We may compute some homotopy groups of $\ul{\ul{\BrPic}}(\Cc)$:

\begin{prop}
  The homotopy groups of $\ul{\ul{\BrPic}}(\Cc)$ are as follows.
  \begin{itemize}
    \item $\pi_1 = \BrPic(\Cc)$.
    \item $\pi_2$ is the set of invertible objects of $\Zc(\Cc)$.
    \item $\pi_3 = k^\times$.
    \item $\pi_i = 0$ for $i \geq 4$.
  \end{itemize}
\end{prop}

There is a bijection between:
\begin{itemize}
  \item Equivalence classes of $G$-graded extensions $\Cc \supset \Dc$ (i.e.\ $\Cc$ is a $G$-graded fusion category with $\Cc_e = \Dc$),
  \item Maps $G \to \ul{\ul{\BrPic}}(\Dc)$
  \item Homotopy classes $BG \to B\ul{\ul{\BrPic}}(\Dc)$.
\end{itemize}
As a map of $1$-skeletons, this is given by sending $\Cc$ to the corresponding $c: G \to \BrPic(\Dc)$.
Extending this to $2$-skeletons corresponds to saying that $c$ is a group homomorphism.
A choice of equivalences corresponds to a choice of equivalences $M_{g,h}: \Cc_g \boxtimes_{\Cc_e} \Cc_h \simto \Cc_{gh}$.
Extending this to $3$-skeletons corresponds to choosing associativity isomorphisms.
Extending this to $4$-skeletons corresponds to checking that the pentagon equation holds (this extension is unique if it exists).

We can tell similar stories for other groupoids.
Maps $BG \to B\ul{\Eq}(\Cc)$ correspond to monoidal $G$-actions on $\Cc$.
Maps $BG \to B\ul{\EqBr}(\Cc)$ correspond to braided $G$-actions on $\Cc$.

\section{(10/14) Constantin Teleman -- Topological Lie Group Actions}

\subsection{Categorical actions of discrete groups}

Last time we talked about actions of discrete groups $G$ on a fusion category $F$.
The automorphism group $\Aut(F)$ is the homotopical group constructed from the following data:
\begin{itemize}
  \item $\pi_0$ is the group of invertible $(F, F)$-bimodules, which is equivalent to the group of graded automorphisms of $Z(F)$.
  \item $\pi_1$ is $Z(F)^\times = \GL_1(Z(F))$.
  \item $\pi_2$ is $\CC^\times$.
  \item The $k$-invariant $k_2$ is a quadratic map $Z(F)^\times \to \CC^\times$.
  \item The $k$-invariant $k_1$ is harder to describe.
\end{itemize}

A $G$-action on $F$ can be encoded by a homotopy class of morphisms $BG \to B\Aut(F)$.
This information can be spelled out in terms of explicit combinatorial / categorical data.

\subsection{Topological actions of higher groups}

We may generalize the above picture to higher groups $G$ and \emph{topological actions}: they are classified by $[BG, B\Aut(\Cc)]$.

An alternative, quasi-equivalent perspective is as follows.
We start with a strict $G$-action and ask for this to factor through homotopy types.
We ``trivialize the action through contractible paths.''
Here are some ways to try to formalize this:
\begin{enumerate}
  \item If the universal cover $\tilde{G}$ is contractible, then a topological action of $G$ is a strict action of $G$ plus fixed-point data for the induced $\tilde{G}$-action (trivializing it).
    This works for $\SL_2(\RR)$, tori, and $\Diff(S^1)$.
  \item A slightly weaker perspective is to consider a differentiable action of $G$ plus a trivialization of the Lie algebra action.
    This can be encoded by an action of the category $\Dsc(G)$.
  \item We may consider the path fibration $\Omega_1 G \to P_1 G \to G$.
    A topological action of $G$ is an action of $\pi_0(G)$ on isomorphism classes of objects of $\Cc$ with a compatible trivialization of $P_1 G$, leading to a topological action of $\Omega G$ on the identity functor $\id_\Cc$.
    This gives an action of $\Omega_1 G$ on every object compatible with all morphisms in $\Cc$.
    We may use this to give an inductive definition of topological $G$-actions (since working with the $\Omega G$-action lowers the category level).
    
    If $\Cc$ is a category or linear dg-category and $G$ is connected, then it suffices to give an $\EE_2$-action of $\Omega G$ on $\id_\Cc$.
    This may also be understood as an $\EE_2$-group homomorphism $\Omega G \to \Aut(\id_\Cc)$.
    Since we want the action to be linear, it suffices to give an $\EE_2$-algebra homomorphism $C_\bullet(\Omega G) \to HC^\bullet(\Cc)$.
    In characteristic zero, the $\EE_2$-algebras here may be spelled out as commutative dg-algebras with compatible Poisson brackets of degree $-1$.
  \item A linear category with topological $G$-action is a bundle of categories with flat connection over $BG$.
\end{enumerate}

\subsection{Examples of topological torus actions}

\begin{ex}
  Let $G = T$ be a torus.
  We may write $G = \exp(\tfr) / \pi$ where $\pi = \pi_1(T)$.
  To specify a topological action of $T$, we need to give a strict $T$-action and a topological trivialization of the $\exp(\tfr)$-action.
  On $\pi \subset \exp(\tfr)$, we get two trivializations: one from strictness of the $T$-actions and one from the trivialization of the $\exp(\tfr)$-action.
  Using one trivialization and then the other gives an $\EE_2$-homomorphism $\pi \to \Aut(\id_\Cc)$.
  Linearizing and writing $\CC[T_\CC^\vee] = \CC\angles{\pi}$, we get $\CC[T_\CC^\vee] = \HH^\bullet(\Cc)$.
  Thinking of $\Cc$ as a coherent sheaf of categories on $\Spec \HH^\bullet(\Cc)$, we get a sheaf on $T_\CC^\vee$.
  Thus a topological action of $\Cc$ corresponds to an ``$\EE_2$ fibering'' $\Cc \to T_\CC^\vee$.
\end{ex}

\begin{ex}
  Let $\Cc$ be a finite semisimple tensor category.
  Then $\HH^\bullet(\Cc)$ is a finite direct sum of copies of $\Cc$.
  A $T$-action corresponds to viewing $\Cc$ as a skyscraper sheaf on $T^\vee$.
\end{ex}

\begin{ex}
  Let $X$ be a compact (Fano) toric variety.
  The Fukaya category $\Fc(X)$ is a $D(\ZZ/2c_1)$-graded category, and $\HH^\bullet(\Fc(X))$ is believed to agree with the quantum cohomology of $X$.
  Let $X = \CC^N \sslash H$, and let $T_\CC = (\CC^\times)^N$.
  Then $T$ acts on $X$ and $\Fc(X)$.
  
  We can construct Givental--Hori--Vafa mirrors as follows.
  Let $T_h^\vee$ be the fiber of $((\CC^\times)^N)^\vee$ over $h \in H^\vee$.
  Let $W= z_1 + \dots + z_n$ on $T_h^\vee$.
  This defines a matrix factorization category $\MF(T^\vee_h; W)$.
\end{ex}

\begin{thm}
  \begin{enumerate}
    \item $\MF(T^\vee_h; W) = \Fc(X; h)$, where $h$ controls the symplectic reduction.
    \item $\MF(T^\vee_h; W)$ is a module category over $\CC[T^\vee_h]$, and Fourier modes of the action correspond to Seidel shift operators.
    \item $\HH^\bullet(\MF(T^\vee_h; W)) = \Jac(W) \simeq QH^\bullet(X; h)$.
  \end{enumerate}
\end{thm}

In the $\ZZ$-graded world, there can't be any interesting $\EE_2$ information here.
In the Givental--Hori--Vafa story, we are periodically graded, so an $\EE_2$-homomorphism between commutative algebras has some extra structure.
This extra structure is captured by the formal germs of $W$ near the critical locus.

\begin{ex}
  At a nondegenerate critical point, we have $\Jac(W) = \CC$, and the germ of $W$ corresponds to an $\EE_2$-structure of a module category over $T^\vee$.
\end{ex}

\begin{ex}
  Let $X = \PP^1 = \CC^2 \sslash \CC^\times$.
  Then $W_h(z) = z + h/z$.
  Critical points occur at $z = \pm \sqrt{h}$.
  As a module category over the Seidel shift operators, $\Fc(\PP^1)$ is not equivalent to $\Vect_{\sqrt{h}} \oplus \Vect_{-\sqrt{h}}$.
\end{ex}

Here's something to note for general use:

\begin{ex}
  If $f: X \to Y$ is a map of smooth dg-manifolds, then an $\EE_2$-structure on the map is a Lagrangian deformation of $\Gamma(df)$ in $T^*X \times T^*Y$.
\end{ex}

\subsection{Categorical Fourier transforms}

On $T \times T^\vee$, there is a \emph{Poincar\'e line bundle} $\Psc$ with flat connection in the $T$ direction.
We may view $T^\vee$ as the moduli of flat line bundles on $T$.
The flat connection on $\Psc$ is given by $d\theta^i \wedge d\theta_i$ where $\theta^i$ and $\theta_i$ are dual bases.

Topologically, we can write $T = B\pi$.
The classifying map $B\pi \times T^\vee \to B\CC^\times$ is bilinear, so it may be delooped to $B^2 \pi \times T^\vee \to B^2 \CC^\times$ classifying ``$B\Psc$''.
This allows us to view $B\Psc$ as a gerbe over $B^2 \pi \times T^\vee$ with flat connection in the $B^2 \pi$ direction.
The flat connection on $B\Psc$ may be written as $\omega^i \wedge d\theta_i$ where $\omega^i \in H^2(BT; \CC)$.

We can use $B\Psc$ to define a Fourier transform from categories over $T^\vee$ to categories with a $B\pi$-action.
The action is given by tensoring with $\Vect$, but it is nontrivial as a homomorphism because it sends $t \in T$ to $\Pc_t \otimes -$ (acting on $\Aut(\id)$).
Flatness of $\Pc$ in the $B\pi$ direction ensures this gives a $B\pi$-action.
It is a good ``exercise'' to unravel this into a map $\pi \to \HH^\bullet(\Cc)$.

\subsection{An example from conformal field theory}

A rational chiral $2$d conformal field theory gives:
\begin{itemize}
  \item A finite semisimple braided tensor category $M$, 
  \item For each $a \in M$, a Hilbert space $\Hc_a$ carrying a projective action of $\Diff(S^1)$,
  \item For each conformal bordism with $N$ inputs and one output and every $a_1, \dots, a_N, b \in M$, a map $\Hom(a_1 \otimes \dots \otimes a_N, b)^\vee \otimes \Hc_{a_1} \otimes \dots \otimes \Hc_{a_N} \to \Hc_{b}$.
\end{itemize}
We'd like this to be a braided $\otimes$-functor, except for the fact that $M$ is a ``topological'' object and the data in $\Hilb$ is ``holomorphic / conformal.''
This functor should be $\Diff(S^1)$-invariant.

We think of the chiral CFT as a boundary theory for a 3d TQFT with $Z(S^1) = M$.
This corresponds to the choice of object $\sum_{a \textrm{simple}} a \otimes \Hc_a^\vee$.

The central charge of the CFT (typically considered modulo $24$) is equal to the central charge of the TQFT.
Next time we'll show that $\Diff(S^1)$ acts on the spaces $\Hc_a$ in an interesting way, but the action on $\Hilb$ is topological.

\section{(10/21) Constantin Teleman -- More on Topological Actions}

\subsection{Conformal field theory continued}

We want to understand the action of a $3$d TQFT $\Tc$ on a 2d holomorphic CFT $\chi$ (coming from a rational conformal field theory).
To get a genuine 2d CFT, we sandwich $\Tc$ between $\chi$ and $\ol{\chi}$ and use a ``$p_1$-structure.''
This produces the 2d CFT $(\chi, \ol{\chi})$.

To specify this information, we need a braided fusion category $\Tc(S^1)$.
We also think of $\chi$ as something like a braided tensor functor $\chi: \Tc(S^1) \to \Hilb$, except that things change holomorphically as $S^1$ acts (instead of changing topologically trivially).
Given $a, b, c \in \Tc(S^1)$, we get Hilbert spaces $\Hc_a$, $\Hc_b$, $\Hc_c$ and maps
\[
  \Hom_{\Tc(S^1)}(a \otimes b, c) \to \Hom(\Hc_a \otimes \Hc_b, \Hc_c).
\]
The object $\oplus_a a \otimes \Hc_a^*$ (in a completion of $\Tc(S^1)$) is a ``chiral algebra.''
It comes with multiplications depending holomorphically on points $z_1, \dots, z_n \in \CC$.

Here are some relevant facts from the theory of conformal surfaces:
\begin{itemize}
  \item The Lie algebra of $\Diff(S^1)$ acts projectively on each $\Hc_a$.
    The \emph{Virasoro algebra} $\Vir$ is a central extension of $\Lie(\Diff(S^1))$ by $\CC$, and $\Vir$ acts (honestly) on each $\Hc_a$ by some $c \in \CC$.
    Actually $c \in \QQ$ (because we are dealing with a rational CFT).
  \item The Virasoro extension corresponds to an extension of groups
    \[
      \begin{tikzcd}
        1 \rar & \widehat{\Diff}(S^1) \rar & \Diff(S^1) \rar & 1.
      \end{tikzcd}
    \]
    Because $\Diff(S^1) \simeq S^1$ (a homotopy equivalence), we have a non-canonical homotopy equivalence $\widehat{\Diff}(S^1) \simeq S^1 \times S^1$.
    In particular, the universal cover of $\widehat{\Diff}(S^1)$ is contractible.
\end{itemize}

If $\Tc$ were oriented, we would get a flat vector bundle $\Tc(\Sigma_g) \to \Mfr_g$.
This is not true in general: instead $\Tc(\Sigma_g) \to \Mfr_g$ has a projectively flat connection.
The Chern slope
\[
  \frac{c}{24} \int_{\Sigma_g} p_1 \in H^2(\Mfr_g)
\]
gives the obstruction to making this flat connection.

``Circles are like surfaces but one dimension less'' -- Constantin Teleman, 2025

So let's think about the universal $S^1$-bundle over $B\Diff(S^1)$.
There's a universal $p_1$ on this, and integrating it (via differential cohomology) gives a class in $H^3(B\Diff(S^1); \ZZ)$.
One may also view $\int p_1$ as the curvature of the $S^1$-gerbe $B\widehat{\Diff}(S^1) \to B\Diff(S^1)$.

\begin{rmk}
  We may view an open subset of $\Mfr_g$ as a quotient of a ``complexification'' of $\Diff(S^1)$, since if we glue $\Sigma^-$ and $\Sigma^+$ along a boundary circle we get
  \[
    T_{\Sigma^- \cup \Sigma^+} \Mfr_g = \Gamma_\hol(\Sigma^-; T\Sigma^-) \backslash \Lie(\Diff(S^1))_\CC / \Gamma_\hol(\Sigma^+; T\Sigma^+).
  \]
\end{rmk}

On $\Tc(S^1)$ we have an $S^1$-action (by rotation) with a $p_1$-anomaly.
This gives a topological $S^1 \times S^1$-action on $\Tc(S^1)$.
In particular, for $a \in \Tc(S^1)$, we get:
\begin{itemize}
  \item A \emph{central charge} $\exp(2\pi i c / 24)$ from the action of the center of $\Diff(S^1)$.
    This is forced on us by the $\SO(3)$ action on the space of TQFTs and encodes the $p_1$-anomaly.
  \item A \emph{conformal weight} $\omega(a)$ from the rotation action.
\end{itemize}

When we go to Hilbert spaces, we have an action of the universal cover of $\widehat{\Diff}(S^1)$ on each $\Hc_a$.
This gives a topological action of $\Diff(S^1)$ on the category spanned by the $\Hc_a$'s.
We may factor this as an action of $\Diff(S^1) / \widetilde{\widehat{\Diff}}(S^1)$ on the category, and the obstruction to trivializing is given by the two numbers above.
Matching the TQFT to the CFT requires us to match the anomalies / numbers above.
These are related by viewing $S^1 \subset \SL_2(\RR) \subset \Diff(S^1)$.

\subsection{Topological group actions via infinitesimal trivializations}

Here's a more concrete approach to topological actions of Lie groups.
This requires us to have differential graded models of our categories, so we can't immediately apply this approach to Fukaya categories.
We'll assume throughout that $G$ is a compact connected Lie group.

Let's consider the classical case of actions on vector spaces.
If we trivialize a $G$-action on a vector space, we don't have access to any interesting information.

Working in the derived world allows for more interesting phenomena.
If $G \curvearrowright (V^\bullet, \partial)$, then we get a map $\Lc: \gfr \to \End^0(V^\bullet)$ with value closed under $\partial$.
To trivialize this, we want a homotopy trivialization $\epsilon: \gfr \to \End^{-1}(V^\bullet)$ such that $[\partial, \epsilon(\xi)] = \Lc_\xi$, $[\epsilon(\xi), \epsilon(\eta)] = 0$, and $\epsilon$ is $G$-equivariant.
This corresponds to an action of the ``differential graded group'' $(G \rtimes \gfr[1], D)$ on $V^\bullet$.
Derived invariants of this are given by $(V^\bullet \otimes \Sym \gfr^*)^G$ with differential $\partial + \xi^a \epsilon(\xi_a)$.
This is the \emph{$G$-equivariant Cartan complex}.

\begin{ex}
  Let $X$ be a $\Cc^\infty$-manifold with $G$-action.
  We get a $G$-action on $(\Omega^\bullet(X), d)$, hence a $\gfr$-action via interior product.
  Taking derived invariants gives the usual Cartan complex computing $H^\bullet_G(X; \RR)$.
\end{ex}

\begin{thm}
  We can form a bundle of complexes on $BG$ with fiber $(V^\bullet, \partial)$.
  The trivialization data gives a connection $\nabla$ on $V^\bullet$ satisfying $\nabla^2 \neq 0$ but $(\nabla + \partial)^2 = 0$.
  On cohomology, this is the Gauss--Manin connection.
  Taking cohomology of $BG$ with coefficients in $V^\bullet$ gives $H^\bullet_G(V^\bullet; \partial)$.
\end{thm}

\subsection{Categorification}

A (dg-)linear $n$-category $\Cc$ has a sequence of Hochschild cohomologies where: 
\begin{itemize}
\item $\EE_0 \HH^\bullet(\Cc) = \End(\Cc)$ is an $\EE_1$-$n$-category.
\item $\EE_1 \HH^\bullet(\Cc) = \Rbf \End(\id_{\Cc})$ (taken in $\EE_0 \HH^\bullet(\Cc)$) is an $\EE_2$-$(n-1)$-category.
\item \dots
\item $\EE_n \HH^\bullet(\Cc) = \Rbf \End(\id_{\Rbf \End(\dots)})$ is an $\EE_{n+1}$-algebra.
In characteristic zero this is a commutative algebra together with a Poisson bracket of degree $(-n)$.
The spectrum of the Chevalley complex here is the moduli stack of formal deformations of $\Cc$.
The tangent space to this moduli stack is $\EE_n \HH^{n+1}(\Cc)$.
\end{itemize}

An infinitesimally trivialized action of $G$ on an $n$-category $\Cc$ is given by a $\Cc^\infty$-action of $G$ on $\Cc$ together with a square
\[
  \begin{tikzcd}
    \gfr \rar["\Lc"] & \EE_n HZ^n(\Cc) \\
    \gfr \rar["\epsilon"] \uar & E_n HCH^{n-1}(\Cc)
  \end{tikzcd}
\]
that commutes and is $G$-invariant.

\begin{ex}
  Let $G$ act on a manifold $X$.
  Then we get an induced action of $G$ on $(\Omega^\bullet(X), d)$.
  An infinitesimal trivialization of this action somehow gives a curved Cartan complex $G \ltimes \Omega^\bullet(X) \otimes \Sym \gfr$.
  More on this next time.
\end{ex}

\section{(10/28) Constantin Teleman -- The Curved Cartan Complex: Examples and Applications}

Recall that we consider an action of a Lie group $G$ on a manifold $X$.
The \emph{curved Cartan complex} is $(\Sym \gfr^* \otimes \Omega^\bullet(X), d_c)^G$ where $d_c = d_\dR + \xi^a \otimes V_a$ where $\{ \xi^a \}$ is a basis for $\gfr^*$ and $\{ V_a \}$ are the corresponding vector fields on $X$.
This $d_c$ satisfies $d_c^2 = \xi^a \otimes \Lc_a$.
In particular, $d_c^2$ vanishes on $G$-invariants.

\subsection{A modification}

Consider the algebra $\Omega^\bullet(X)$.
We may consider the crossed product
\[
  (G \ltimes \Omega^\bullet(X), d_c)
\]
where $d_c$ is given by the formula mentioned before.
Really here we replace $G$ by distributions on $G$, i.e.\ the algebra $(\Cc^{-\infty}(G), \star)$.
This satisfies the relations of a \emph{curved algebra}, i.e.\ we have $d_c^2 = [W, -]$ for some $W$ satisfying $[W, W] = 0$.
Here we may take $W = \xi_a \delta_1 \otimes \xi^a \otimes \id_{\Omega^\bullet(X)}$.

More generally, suppose $A$ is a dg-algebra with smooth $G$-action and infinitesimal trivialization, i.e. a dgla / $\LL_\infty$ morphism
\[
  \begin{tikzcd}
    \gfr \rar & \mathrm{HZ}^1(A) \\
    \gfr \uar[equal] \rar & \mathrm{HCH}^0(A) \uar["\beta", swap]
  \end{tikzcd}
\]
where the lower $\gfr$ has trivial multiplication and $\beta$ is the Hochschild differential.
Then we may form a curved Cartan complex $\CCC(A) = G \ltimes \Sym \gfr^* \otimes A$ (with suitable differential).

\begin{rmk}
  The category of modules over $G \ltimes \Sym \gfr^*$ with $W = \xi_a(\delta_1) \otimes \xi^a$ is a tensor category (constructed via the Hopf structure on $\Cc^{-\infty}(G)$).
  ``Coupling'' $A$ to the CCC corresponds to viewing the category of $A$-modules as a module category over this tensor category.
\end{rmk}

Curved modules over $\CCC(A)$ are the derived $G / \hat{G}$-invariants in $A$-mod, where $\hat{G}$ is the formal completion of $G$ at its Lie algebra.

\begin{rmk}
  Completing at $0 \in \gfr$ makes the CCC into $G \ltimes (\wedge^\bullet \gfr, \partial) = (\Omega^\bullet_{-\infty}(G), d)$.
  Categories with an action of this correspond to categories with infinitesimally trivialized $G$-action.
\end{rmk}

\begin{ex}
  Let $G$ be a torus.
  Then $\Cc^{-\infty}(T) = \prod_{\lambda \in \weight(T)} \CC_\lambda$.
  Thus $\Spec T \ltimes \Sym \tfr^* = \oplus_{\lambda \in \weight(T)} \tfr_\lambda$.
  The corresponding $W$ sends $\xi \in \tfr_\lambda$ to $\lambda(\xi)$.
  There are equivalences between the following categories:
  \begin{itemize}
    \item The matrix factorization category $\MF(\oplus_{\lambda \in \weight(T)} \tfr_\lambda, W)$,
    \item The category of $\CC[\tfr]$-modules,
    \item The category of $H^*(BT)$-modules,
    \item The category of derived $T / \hat{T}$-invariants in $\dgVect$, and
    \item The category of derived local systems on $BT$ with flat connection.
  \end{itemize}
\end{ex}

\subsection{Examples from symplectic geometry}

Suppose $X$ is a symplectic manifold with Hamiltonian $G$-action and moment map $\mu: X \to \gfr^*$.
Let $A$ be a deformation quantization of $\Cc^\infty(X)$ over $\CC((h))$.
If $G$ is compact, we may take the deformation quantization to be $G$-equivariant in a strong sense:
\[
  [\mu_h(\xi), a] = \Lc_\xi(a)
\]
for a quantum moment map $\mu_h: U\gfr \to A$.
This defines an infinitesimal trivialization of the $G$-action on $A$ via the morphism of dglas 
\[
  \begin{tikzcd}
    \gfr \rar & \Der(A) \\
    \gfr \uar[equal] \rar["\mu_h"] & A \uar["{a \mapsto [a, -]}", swap].
  \end{tikzcd}
\]
The corresponding CCC is $G \ltimes (\Sym \gfr^* \otimes A)$ with $W = (\xi_a \delta_1 \otimes \id_A + \id_G \otimes \mu_h(\xi_a)) \otimes \xi^a$.

The CCC here is an ``integration of the $\gfr$-action on $A$.''
More precisely, we may view $A$ as a Harish-Chandra $(\gfr \oplus \gfr, G)$-module, where the two copies of $\gfr$ come from left and right actions.
If $A$ is a free $U\gfr$-module, the CCC is a maximally integrable $G \times G$-module.
More generally, the CCC is a derived analogue of a maximally integrable module.
We may write this as $H^\bullet(\hat{G} \backslash G / \hat{G}; A)$.

\begin{ex}
  In the abelian case, we get $\oplus_\lambda \CC[\tfr]_\lambda \otimes A$.
  The superpotential here is $W: \xi \mapsto (\lambda - \mu)(\xi)$.
  The corresponding matrix factorization category is equivalent to the category of $A / (\lambda - \mu)$-modules.
\end{ex}

\begin{ex}
  Consider $X = \gfr^*$ as a Poisson manifold, with moment map given by the identity.
  The quantization is $U\gfr$ and the quantum moment map is still the identity.
  The CCC is $\Gfr \ltimes (U\gfr \otimes \Sym \gfr^*)$ with $W = (\xi_a \delta_1 \otimes \id_{U\gfr}) \otimes \xi^a$.
  We end up seeing that the $G \times G$-integrable part of $U\gfr$ is $\oplus_V \End(V)$, where $V$ ranges over the finite-dimensional $G$-representations.
  Modules over this CCC are representations of $G$.
  Conceptually, we may view $U\gfr$-modules as $\Vect^{\hat{G}}$, and we have $(\Vect^{\hat{G}})^{G/\hat{G}} = \Vect^G$.
\end{ex}

\subsection{A surprisingly nontrivial example}

\begin{ex}
  Let $A = \CC$.
  An infinitesimal trivialization corresponds to a $G$-invariant $\LL_\infty$-map
\[
  \begin{tikzcd}
    \gfr \rar & 0 \\
    \gfr \uar[equal] \rar["W"] & \CC \uar["0"]
  \end{tikzcd},
\]
  and any $G$-invariant $W: \gfr \to \CC$ gives an $\LL_\infty$-map.
  (We may have to collapse the grading mod $2$ to make this work.)
\end{ex}

\begin{thm}[Freed-T]
  Let $W = Q: \gfr \to \CC$ be the quadratic invariant form.
  Then:
  \begin{enumerate}
    \item $\MF(\gfr, W)^G \simeq \Rep(G)$.
    Moreover, representations are supported at their coadjoint orbits.
  \item Consider the coadjoint action of $LG$ on $L\gfr^*$ at a nonzero level $\tau$ (equal to $h$ plus a dual Coxeter number).
    Also let $G$ act on itself by conjugation.
    Then $\MF^\tau(G; Q)^G \simeq \Rep(LG; h)$, and the representations are supported on the ``correct'' conjugacy classes.
  \end{enumerate}
\end{thm}

The twisted matrix factorization categories above correspond to modules over an Azumaya algebra.

This gives a curious duality between $\Vect^G$ and $(\Vect, Q)^{G/\hat{G}}$.
Both give the same TQFTs as $\Rep(G)$, i.e.\ topological Yang--Mills in two dimensions.
The former ``counts'' via an integral over a moduli space of flat connections with Riemannian metrics.
The latter ``counts'' by viewing $Q$ as a class in $H^4(BG; \RR)$ and sending $\Sigma$ to
\[
  \int_\Sigma \ev^* Q \in H^2(\Bun_G(\Sigma)).
\]

T's former student Kiran Luecke showed that this categorifies the Kirillov character formula for compact groups.
More precisely, he defined a Chern character of $\MF$ and showed that this has two limits: one via characters of irreps and the other as an integral over coadjoint orbits.
Kirillov showed that $\exp(\chi_V \cdot \sqrt{\dim})$ is an integral over the orbit of $V$.
For real semisimple groups and discrete series we get a Harish-Chandra formula.
For loop groups of compact groups we get a Frenkel integral formula (though this is not written down.

\begin{qn}
  For real semisimple groups and principal series representations, what do we get?
  This problem is especially interesting as we approach singular values.
\end{qn}

If we write $\omega = \int_\Sigma Q \in H^2(\Bun_G(\Sigma), \RR)$, we get formulas for
\[
  \int_{\Bun_G(\Sigma)} \exp(\omega) \cdot (\textrm{point operators}).
\]
What happens if, instead of dealing with $Q$, we consider other classes in $H^\bullet(BG; \RR)$?

\begin{thm}
  Let $\phi$ be an invariant function on $\gfr$, and consider a formal deformation $Q + t\phi$ (with deformation parameter $t$).
  Then $\MF(\gfr; Q + t\phi)^G$ is a semisimple category supported at the $t$-perturbations of coadjoint orbits.
  This controls the TQFT which computes
\[
  \int_{\Bun_G(\Sigma)} \exp\left(\int_\Sigma \ev^*(Q + t\phi)\right) \cdot (\textrm{point operators}).
\]
  We can get a Verlinde-style formula out of this.
  The structure constants for the relevant Frobenius algebra may be computed by multiplying the volumes of perturbed classes and the Hessian determinant of $W$ at critical points.
  The same holds for loop groups and $K$-theory.
\end{thm}

\begin{ex}
  For $\xi \in \gfr$, we may take $\phi = \tr_V(\xi(\log \xi - 1))$.
  Equivariant integration over the universal bundle $P \times_G V$ on $\Sigma \times \Bun_G(\Sigma)$ gives an index bundle along $\Sigma$.
  We get GLSM / Gromov--Witten invariants out of this.
  There's also a K-theoretic version of this involving dilogarithms.
\end{ex}

The upshot is that the CCC lets us do some computations of gauged $A$-models, though we need to know the superpotential $W$.
In general $W$ may be multivalued.

\begin{qn}
  Is there a good theory of matrix factorizations for multivalued superpotentials?
\end{qn}

\section{(11/4) Zechen Bian -- The BFM Space}

Our goal today is to understand the BFM space $\BFM(\check{G})$.

\subsection{Descriptions of the BFM space}

There are many descriptions of this space:
\begin{enumerate}
  \item $\BFM(\check{G}) = \Spec H_*^{G_\Osc}(\Gr_G)$, where $\Gr_G$ is the affine Grassmannian (and the multiplication is given by convolution).
    (There's also a $K$-theoretic version.)
  \item $\BFM(G)$\footnote{Not $\BFM(\check{G})$!} is $T^*_\reg(G) \sslash_0 G$, where $T^*_\reg$ indicates the open subscheme of the cotangent bundle consisting of regular elements of $\gfr$.
    Recall that $x \in \gfr$ is regular iff $\dim \zfr(x) = \rk \hfr$, i.e.\ the centralizer of $x$ has minimal rank.
    For $G = \GL_n$, regular elements consist of matrices such that every eigenvalue has geometric multiplicity at most one.
    The moment map $T^*_\reg G \to \gfr^*$ is given by $(\xi, g) \mapsto \Ad_g(\xi)$.
  \item Consider the \emph{universal centralizer}
    \[
      C_{G,\gfr} = \bset{(g, \xi) \in G \times \gfr}{\xi \textrm{ regular and } g \in Z(\xi)}.
    \]
    Let $\Zfr_{G,\gfr} = C_{G,\gfr} \sslash G$.
    Then $\BFM(G) \cong \Zfr_{G,\gfr}$.
  \item Let $\mathring{\Bfr}_\gfr^G = \Bl_{D_T \times D_\tfr}(T \times \tfr) \setminus E_{D_T \times D_\tfr}$, where:
    \begin{itemize}
      \item $D_T$ is the union in $T$ of the kernels of all roots.
      \item $D_\tfr$ is defined similarly but lives in $\tfr$.
      \item $E_{D_T \times D_\tfr}$ is the strict transform of $D_T \times D_\tfr$.
    \end{itemize}
    Define $\Bfr_\gfr^G = \mathring{\Bfr}_\gfr^G \sslash W$.
\end{enumerate}

\begin{thm}[BFM]
  There are isomorphisms
  \[
    \Spec H^{G_\Osc}_*(\Gr_G) \cong \Zfr_{\check{\gfr}}^{\check{G}} \cong \Bfr_{\check{\gfr}}^{\check{G}}.
  \]
  These are compatible with the natural maps to $\tfr / W$.
  The natural symplectic structures on the first two descriptions agree.
\end{thm}

In the $K$-theoretic case, we have an isomorphism $\Spec K_*^{G_\Osc}(\Gfr_G) \cong \Bfr_G^{\check{G}}$.

Recall that $\Gr_G$ is defined by $\Gr_G(R) = G(R((t))) / G(R[[t]])$.
There is a natural $G_\Osc$-action on this by left multiplication.
We also have a $\GG_m$-action by rotating the parameter $t$.
This allows us to consider $H^{G_\Osc \ltimes \GG_m}(\Gr_G)$ as a noncommutative algebra over $H^{G_\Osc \ltimes \GG_m}(\pt) = \CC[\tfr / W]\angles{\hbar}$.
This gives a deformation of $H^{G_\Osc}_*(\Gr_G)$ and hence a natural Poisson structure on $\Spec H^{G_\Osc}_*(\Gr_G)$.

Here's an outline of the proof of the above theorem in the $K$-theory case:
\begin{enumerate}
  \item We can write $K^G(\Gr_G) = K^T(\Gr_G)^W$ and reduce to studying $K^T(\Gr_G)$ over $T / W$.
  \item For $g \in T$ regular, we have $(\Gr_G)^g = (\Gr_G)^T \cong \Gr_T$.
    Thus, by localization, the module $K^T(\Gr_G)_g$ is free of rank $1$.
    Note that $g$ is regular if and only if $g$ is not in the kernel of any roots.
  \item A computation shows that $\Spec K^T(\Gr_G)|_{T_\reg / W} \cong \Bfr_G^{\check{G}} |_{T_\reg / W}$.
    In the homology case, we can do this by writing $S_\alpha = \CC[T \times \tfr][(e^\alpha - 1) / \alpha, 1 / \beta]$ for $\beta \neq \pm \alpha$.
    Then $\Bfr_\gfr^G = \Spec \cap_\alpha S_\alpha$.
  \item To deal with the codimension-1 case, look at ``almost regular'' elements and do an $\slfr_2$-computation.
\end{enumerate}

\subsection{Role of BFM}

Consider a topological action of a group $G$ on a category $\Cc$.
This corresponds to an $\EE_2$-algebra homomorphism $C_*(\Omega G) \to HC(\Cc)$.
We may view $Z = \Spec H_*(\Omega G)$ as a Lagrangian in $\BFM(G^\vee)$.
Having $Z$ is not enough for computing fixed-point categories and other things of interest.
Instead we want to consider the whole $2$-category $\KRS(\BFM(\check{G}))$.
Gauged $2$d TQFTs will correspond to boundary theories of pure $3$d gauge theory, i.e.\ objects of this $\KRS(\BFM(\check{G}))$.
The object $Z$ corresponds to the regular / Dirichlet boundary condition.

\begin{ex}
  If $\gfr = \glfr_n$, then $Z$ is the zero fiber of $\Zfr_{\gfr}^G \to \tfr / W$.
  This fiber consists of regular nilpotent elements.
\end{ex}

\begin{ex}
  If $G$ is a torus, then $\BFM(T) = T^* \check{T} \cong \check{\tfr}^* \times \check{T}$.
  We may write $\Omega T \simeq \coweight(T) \simeq \weight(\check{T})$.
  Thus $H_*(\Omega T) = \CC[\weight(\check{T})]$, so $Z$ corresponds to the zero section $\check{T}$.
  The Neumann boundary condition should correspond to the vertical fiber $T^*_1 \check{T}$.
\end{ex}

\begin{rmk}
  A paper of Xin Jin shows that the wrapped Fukaya category of $\BFM(G)$ is equivalent to $\Coh(\check{T} \sslash W)$.
  This should be some sort of 2d reduction of the 3d mirror symmetry story.
\end{rmk}

\section{(11/13) Constantin Teleman -- The Toda Space and 3D Gauge Theory}

\subsection{Examples of Toda spaces}

We start with some examples:

\begin{ex}
  Let $T = S^1$.
  The Toda space is $\Tc(T) = T^*T_\CC^\vee$.
  There is an integrable system $\Tc(T) \to \tfr$, with unit section giving the Neumann boundary condition.
  The zero section of $\Tc(T) \to T_\CC^\vee$ is the Dirichlet boundary condition.
\end{ex}

\begin{ex}
  Let $G = \SU(2)$.
  The corresponding Toda space is $(T^* T^\vee_\CC) / W$ blown up at $(1, 0)$.
  There's a natural map to $\tfr / W$, and the unit section is the Neumann boundary condition.
  The Dirichlet condition corresponds to the strict transform of the zero section.
\end{ex}

\subsection{Gauge theory}

If $\Cc$ is a category with topological $G$-action, we get an $\EE_2$ map $C_*(\Omega G) \to \HH^*(\Cc)$.

Kontsevich formality gives an equivalence 
\[
HCH^*(\Cc^\infty(X)) \simeq (\Gamma(X, \wedge T_X), \{-,-\}) \simeq \Cc^\infty(T^*X[1]),
\]
where $T_X$ is concentrated in degree $1$.
This continues to higher $\HH^*$, giving $\EE_2 \HH^* \simeq \Gamma(X; \widehat{\Sym T_X})$ where $T_X$ is now concentrated in degree $2$ (and we consider both sides with the degree $-2$ Poisson bracket).
Applying this to the above, we get $\EE_2 \HH^*(H_*(\Omega G)) = \widehat{H_*^G(\Omega G)} \to \HH_G^*(\Cc)$.
  
\begin{ex}
Suppose $\Cc = \Fc(X)$ is a Fukaya category with $X$ a compact symplectic $G$-manifold.
We expect that $\HH^* = \HH_* = QH^*(X)$.
Results of Kirwan show that $QH^*_G(X)$ is a finite free $H^*(BG)$-module.
\end{ex}

\begin{conj}
  The category $\Cc_0 = \Cc$ has a deformation $\Cc_{\tfr / W}$ over $\tfr / W$ which lives over $\Spec \HH^*_G(\Cc) \subset \Tc(G)$.
\end{conj}

Our above discussion gives this deformation in a formal neighborhood.
The formal deformation theory determines $\Cc$ near $D$.

\begin{ex}
  If $\Cc = \Fc(X)$ as before, then we'd see that $\HH^*_{\tfr/W}(\Cc_{\tfr / W}) = \HH^*_G(\Cc) = QH^*_G(X)$ as a finite module over the polynomial ring $H^*(BG)$.
\end{ex}

\begin{rmk}
  When $X$ is Fano, Pomerleano showed that $QH_G^*(X)$ is finite over $H^*(BG)$ and is an $\EE_2$-algebra over the $\EE_3$-algebra $\CC[\Tc(G)]$.
\end{rmk}

\begin{rmk}
  From this we get a connection $\nabla$ on a sheaf of categories.
  This $\nabla$ has components in $\HH^i \otimes \Omega^{2-i}$, so its curvature has components in $\HH^i \otimes \Omega^{3-i}$.
\end{rmk}

Here are some things we can compute with our boundary conditions:
\begin{itemize}
  \item The Dirichlet boundary condition computes the original / underlying category.
  \item The Neumann boundary condition computes the fixed point category: $\Hom(N; \Cc_{\tfr/W}) = \Cc^{G_\top}$.
    In particular we have $\Osc_N \otimes_{\Osc_\Tc} \HH_G^*(\Cc^{G_\top})$.
    This calculation may be performed via the curved Cartan complex if we know $\Cc_{\tfr/W}$ in a formal neighborhood of $N$.
\end{itemize}

There were some computational examples for $\PP^1$ with an $S^1$-action and an $\SO(3)$-action, but I didn't follow them very well.
It seems that some computations (especially in the derived world) are fairly difficult / unclear.

\subsection{The cross-foliation in the nonabelian case}

Classically, we consider $T \curvearrowright X$ with moment map $\mu: X \to \tfr^*$.
The quantum cohomology of the fiber $QH^*$ will depend on the parameter $\tau \in \tfr^*$.

In the nonabelian case of $G \curvearrowright X$, when we change the value of $\mu: X \to \gfr^*$, we really have to work with fibers over orbits $\Oc \subset \gfr^*$.
The deformation of $X$ is a reduction of $(X \times F_X) \sslash G$.

\begin{thm}[Whittaker presentation of $\Tc(G)$]
  We may write $\Tc(G) = T^*_\reg(G_\CC^\vee) \sslash \Ad_{G^\vee} = N \bsslash_\chi T^*(G_\CC^\vee) \sslash_\chi N$.
  At regular nilpotent characters, the Lagrangian fibers of this over $N \backslash G_\CC^\vee / N$ are flag varieties.
\end{thm}

\section{(11/18) Erik Herrera -- Deligne's Conjecture: An Operadic Story}

Our goal today is to build up to a discussion of Deligne's conjecture.

\subsection{Operads in vector spaces}

To discuss Deligne's conjecture, we first need to talk about operads.
These give a way of encoding certain algebraic structures appearing in TQFT.

\begin{dfn}
  An \emph{operad} consists of:
  \begin{itemize}
    \item A collection of vector spaces $P(n)$, $n \geq 0$,
    \item An action $S_n \curvearrowright P(n)$ for all $n$,
    \item An identity element $\id \in P(1)$, and
    \item Composition laws $m_{n_1,\dots,n_k}: P(k) \otimes P(n_1) \otimes \dots \otimes P(n_k) \to P(n_1 + \dots + n_k)$
  \end{itemize}
  all subject to certain associativity / unitality relations (which we will omit).
\end{dfn}

\begin{ex}
  Let $V$ be a vector space.
  We can define an operad $\End(V)$ with $\End(V)(n) = \Hom(V^{\otimes n}, V)$.
  Here $m_{(n_1, \dots, n_k)}$ takes multilinear maps of arities $n_1, \dots, n_k$ and combines them into a multilinear map of arity $n_1 + \dots + n_k$ using an auxiliary operation of arity $k$.
\end{ex}

\begin{ex}
  There is a \emph{commutative operad} $\Com$ with $\Com(n) = k$ for all $n$, where $S_n$ acts trivially.
  This captures the fact that there is only one way to multiply elements $a_1, \dots, a_n$ in a commutative algebra.
\end{ex}

\begin{ex}
  There is an \emph{associative operad} $\Assoc$ with $\Assoc(n) = k^{n!}$ for all $n$, where $S_n$ acts by the regular representation on $\Assoc(n)$.
  This captures the fact that there are $n!$ different ways to multiply elements $a_1, \dots, a_n$ according to how we order them.
\end{ex}

Note that operads don't encode all of the algebraic structures we might be interested in: for example, it's hard to discuss invertibility.

One can define a natural notion of ``morphism of operads.''
There is a morphism of operads $\Assoc \to \Com$ corresponding to the fact that any commutative algebra is associative.
Of course, making this last statement precise requires us to find some way of connecting our operads to genuine algebras.

\begin{dfn}
  An algebra over an operad $P$ consists of a vector space $A$ and a collection of maps $f_n: P(n) \otimes A^{\otimes n} \to A$ such that:
  \begin{itemize}
    \item $f_n$ is $S_n$-equivariant.
    \item $f_1(\id_P \otimes a) = a$ for all $a \in A$.
    \item the maps $f_n$ commute with composition
  \end{itemize}
\end{dfn}

In other words, a $P$-algebra structure on $A \in \Vect$ is an operad morphism $P \to \End(A)$.

\subsection{Operads in topology}

The definition of operads makes sense (with some mild changes) in any symmetric monoidal category.
In particular, we can do so in the category of topological spaces:

\begin{dfn}
  A \emph{topological operad} $P$ consists of:
  \begin{itemize}
    \item A space $P(n)$ for all $n \geq 0$,
    \item A continuous action of $S_n$ on $P(n)$, 
    \item An identity element $\id \in P(1)$, and
    \item Composition laws $m_{n_1,\dots,n_k}: P(k) \times P(n_1) \times \dots \times P(n_k) \to P(n_1 + \dots + n_k)$
  \end{itemize}
  satisfying composition axioms.
\end{dfn}

\begin{ex}
  For any compact topological space, we can make sense of $\End(X)$: $\End(X)(n) = \Map(X^{\times n}, X)$ with the compact-open topology.
\end{ex}

\begin{ex}
  There is a \emph{little $d$-disks operad} $C_d$ defined as follows.
  Let $D$ be a $d$-dimensional closed disk.
  Then $C_d(n)$ is the space of embeddings $D^{\sqcup n} \to D$.
  Composition is given by nesting disks.
  This is useful for understanding homotopy groups.
\end{ex}

For $d = 1$, the operad $C_1$ controls embeddings of an interval into a larger interval.
In particular, based loop spaces $\Omega X$ are $C_1$-algebras.
We may generalize this further:

\begin{thm}[May recognition]
  Any $k$-fold loop space $\Omega^k X$ is a $C_k$-algebra.
  Conversely, any ``grouplike'' $C_k$-algebra has the weak homotopy type of a $k$-fold loop space.
\end{thm}

\subsection{DG-operads}

A differential graded (or dg-)operad $P$ is an operad in chain complexes.
That is, each $P(n)$ is a chain complex.
We think of the $P_0(n)$ as controlling arity $n$ operations, while the other $P_i(n)$ control (higher) homotopies.

Given a topological operad $P$, we can produce a dg-operad $\Chain(P)$ with $(\Chain(P))(n) = \Chain(P(n))$.
We will model ``chains'' here by rectangular chains, i.e.\ $\Chain_k(X)$ is the group of finite $\ZZ$-linear combinations of maps $[0, 1]^k \to X$.
With this model, there is a natural map
\[
  \bigotimes_i \Chain(X_i) \to \Chain\Big(\prod_i X_i\Big)
\]

\begin{ex}
  We define the $E_k$ operad as $E_k = \Chain(C_k)$.
\end{ex}

\begin{dfn}
  Let $P$ be a dg-operad.
  Then $H_*(P)$ is also a dg-operad, where $(H_*(P))(n) = H_*(P(n))$ with all differentials given by zero.
\end{dfn}

\begin{ex}
  We have $H_*(E_1) = \Assoc$.
  Morally, this means that $E_1$ is a ``homotopically freer'' version of $\Assoc$.
\end{ex}

We'd like to understand the relationship between taking homology of operads and taking homology of algebras.

\begin{thm}[Transfer]
  Suppose an operad $P$ is freely generated in arity $\geq 2$.
  If $A$ is a $P$-algebra in complexes, then $H_*(A)$ has a canonical $P$-algebra structure.
\end{thm}

\begin{thm}
  Suppose some reasonable assumptions hold.
  If $A$ is an algebra in chain complexes over a dg-operad $P$, then $H_*(A)$ is an algebra over $H_*(P)$.
\end{thm}

However, the relationship between $P$ and $H_*(P)$ is not clear in general!

\subsection{Deligne's Conjecture}

Let $A$ be an associative algebra.
Then the Hochschild cohomology $\HH^*(A)$ has a Gerstenhaber structure: there is a cup product on cochains and a bracket $[-,-]$ satisfying the Leibniz rule.
This bracket is defined via a mysterious formula: for $\phi \in HC^k(A)$ and $HC^\ell(A)$, we let
\[
  (\phi \circ \psi)(a_1 \otimes \dots \otimes a_{k+\ell-1}) = \sum_i (-1)^{i(\ell-1)} \phi(a_1 \otimes \dots \psi(a_{i+1} \otimes \dots \otimes a_{i+\ell}) \dots \otimes a_{k+\ell-1})
\]
and
\[
  [\phi, \psi] = \phi \circ \psi - (-1)^{(k-1)(\ell-1)} \psi \circ \phi.
\]

The reason this interests us is because the Gerstenhaber operad is isomorphic to $H_*(E_2)$!
In fact, $C_2(2) \simeq S^1$, so $H_*(E_2(2))$ has one generator in degree $0$ and one generator in degree $1$.
The generator in degree $0$ corresponds to the cup product, and the generator in degree $1$ corresponds to the bracket.
In particular, we see that $\HH^*(A)$ is an algebra over $H_*(E_2)$.

\begin{conj}[Deligne]
  The Gerstenhaber structure on $\HH^*(A)$ comes from an $E_2$ structure on $C^*(A, A)$.
\end{conj}

This is somewhat surprising, and can be thought of as constructing a bulk theory to match our boundary theory $A$.
More precise versions of this conjecture have been proven.
Let's try to build up to one of these.

\begin{dfn}
  The \emph{Swiss cheese operad} $SC_d$ consists of the topological spaces $SC_d(n,m)$ of embeddings of $m$ semidisks and $n$ disks into a standard semidisk in $\RR^{d+1}$.
  The semidisks must all appear on the boundary of the large semidisk.
  (For $n = m = 0$, we set $SC_d(0, 0) = \varnothing$, and for $n = 0$, $m = 1$, we set $SC_d(0, 1) = \pt$.)
\end{dfn}

This is not a genuine operad but rather a ``colored operad,'' with two different kinds of objects.
Note that $SC_d(n, 0) \simeq C_{d+1}(n)$ and $SC_d(0, m) \simeq C_d(m)$, so $SC_d$ combines features of $C_d$ and $C_{d+1}$.
An algebra over $SC_d$ consists of a $C_d$-algebra $A$, a $C_{d+1}$-algebra $B$, and an action of $B$ on $A$.
(We may take this as a \emph{definition} of actions of $C_{d+1}$-algebras.)

\begin{thm}[Generalized Deligne conjecture]
  For every $C_d$-algebra $A$, there exists a universal $C_{d+1}$-algebra $\Hoch(A)$ such that $\Hoch(A) \curvearrowright A$.
  An action of $B$ on $A$ is equivalent to a homomorphism $B \to \Hoch(A)$.
\end{thm}

One can show (but it's hard) that $\Hoch(A) = \End(A)$ as $E_d$-modules.

\section{(11/25) Constantin Teleman -- Summary of Toda Spaces}

\subsection{The torus case}

The Toda space of a torus $T$ is $T^*T_\CC^\vee$.
We think of this as fibered over $\tfr = \Spec H^*(BT)$.
Categories with topological $T$-action correspond to categories with $\HH^*$ given by an $\EE_2$-algebra over $T^* T_\CC^\vee$.
The unit section of $T^*T_\CC^\vee \to \tfr$ corresponds to the Neumann boundary condition, i.e.\ the trivial $\onebb$.
The fiber $T_\CC^\vee$ corresponds to the Dirichlet boundary condition, i.e.\ the regular representation $T \curvearrowright \Loc(T)$.
These intersect in an ``electromagnetic duality point'' (corresponding to a categorical Fourier transform).

Points on $T_\CC^\vee$ give other sections corresponding to topological actions of $T$ on $\Vect$.
These are the analogues of irreducible $1$-dimensional representations.

If we try to relate this to toric mirror symmetry for $\PP^1$, we look at categories corresponding to the Lagrangian $\Gamma_{dW}$ for $W = Q(z + z\inv)$.

\subsection{The nonabelian case}

For the adjoint case, we look at $T^*_\reg(G^\vee) \sslash_{\Ad} G^\vee$.
This fibers over $(\gfr^\vee)^* \sslash G^\vee \cong \tfr / W \cong \Spec H^*(BG)$.
There's also a map to $G^\vee /_\Ad G^\vee = T^\vee / W$ which can help us reduce to $T \subset G$.
The Fourier modes on $T^\vee$ correspond to Seidel shift operators on Fukaya categories / Hochschild cohomology.

There's also a Whittaker picture.
Fix $N^\vee \subset G^\vee_\CC$.
Our Toda space is also equivalent to $N \bsslash_\chi T^* G_\CC^\vee \sslash_\chi N$, where $\chi: \nfr \to \CC$ is a regular nilpotent.
This maps to $N \backslash G_\CC^\vee / N$, which contains $w_0 T_\CC^\vee$ as an open stratum.
There fibers only over $w_0 w_L\inv$ where $w_L$ is the longest element of some Levi.
For these the relevant torus is dual to the maximal torus in the Levi of $G$.

The fibers of this Whittaker map are Lagrangian and affine, and they correspond to mirrors of flag varieties.
They agree with Rietsch's mirrors in appropriate coordinates.

\begin{conj}
  There is a $H^*(BG)$-action with shift operators coming from projection to $G^\vee / G^\vee$.
\end{conj}

The inclusions $N^\vee \subset G^\vee$ give a family of irreducible categorical representations in the Whittaker picture.
This is a categorical version of Borel--Weil.

\begin{qn}
  Find an intrinsic meaning / algebraic construction of this irreducible family of categorical representations.
\end{qn}

\subsection{$K$-theory version}

There are two approaches to making this story work in $K$-theory:
\begin{itemize}
  \item A quantum group approach produces $J$-functions of flag varieties via a ``Whittaker construction.''
    These $J$-functions are ``solutions to a quantum $D$-module.''
  \item A loop group approach constructs the ``correct space'' $\Spec K_*^G(\Omega G) = T_\CC^\vee \times T_\CC / W$.
\end{itemize}

\begin{qn}
  How can we reconcile these via $QK^*$ of flag varieties?
  T thinks that there is a Poisson Lie group structure on the double $D(G^\vee)$ corresponding to a quantum group.
\end{qn}

\subsection{An example relating Toda and CCC}

Let $\tau$ be our variable on $\tfr$.
Let's consider $\Vect$ with Dijkgraaf--Witten $G$-action
\[
  W(\tau) = \frac{h}{2} \tau^2 + \epsilon \phi(t) \in H^4(BT; \ZZ) \cong H^3(BG; \Osc^\times).
\]
Intersections with the unit section are $h\inv(\weight(G)) + O(\epsilon)$.
This gives a semisimple fixed-point category and hence a (colimit of) 2d TQFTs.

The topological interpretation of this corresponds to integration on a moduli space of flat $G$-connections, i.e.\
\[
  \int_{\Bun_G(\Sigma)} \exp\Big(\ev^* W\Big)
\]
The partition function of this TQFT was computed by Witten in 1992 (as a limit of 2d Yang--Mills).

For the curved Cartan complex picture, we consider the $L_\infty$ map:
\begin{itemize}
  \item in degree $0$, the zero map $\tfr \to HZ^1$
  \item in degree $1$, $W: \tfr \to \HH^{\textrm{even}}$
\end{itemize}
Here the superpotential $t_a(\delta_1) \otimes \delta^a + W(\tau)$ on $T \ltimes \tfr$ unravels $W(\tau) + \lambda(\tau)$ on $\sqcup_\lambda \tfr$.
We end up getting the same story with infinitely many intersection points as in the Toda space computation.

\subsection{A final example}

Let's consider the regular representation $T \curvearrowright \Loc(T) = \Coh(T^\vee)$ (i.e.\ the Dirichlet boundary condition).
In the Toda picture, there is one intersection point between the Dirichlet and Neumann boundary conditions.
This corresponds to writing $\Loc(T)^T = \Vect$.

Let's try to do this with CCC.
We need an algebraic presentation of the $T$-action here.
Consider the Poincar\'e bundle $P \to T \times T_\CC^\vee$, which is holomorphic over $T_\CC^\vee$ and flat over $T$.
The bundle $P$ is multiplicative in $t \in T$, i.e.\ we have $P_t \otimes P_{t'} \simeq P_{t t'}$ as bundles on $T_\CC^\vee$.
The $T$-action here is no longer trivial.
The CCC picture here corresponds to $T \ltimes D(\CC\angles{\pi_1(T)}\dMod)$ with superpotential $\tau_a(\delta_1) \otimes \tau^a$.
Passing to fixed points gives $(\CC[\tfr^\vee \times \tfr], W = \sum_a \tau_a \tau^a)\dMod$, which is just $\Vect$ by Kn\"orrer periodicity.

\end{document}

